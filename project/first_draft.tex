\documentclass[spanish, fleqn]{article}
\usepackage[spanish]{babel}
\usepackage[utf8]{inputenc}
\usepackage{amsmath}
\usepackage{amsfonts,txfonts}
\usepackage{mathrsfs}
\usepackage[colorlinks, urlcolor=blue]{hyperref}
\usepackage{fourier}
\usepackage[top = 2.5cm, bottom = 2cm, left = 2cm, right = 2cm]{geometry}
\usepackage{graphicx}




\title{Resolución de ecuaciones de Euler-Lagrange por métodos espectrales}
\author{Martín Villanueva}
\date{29 de Noviembre de 2016}

\begin{document}
\maketitle



\section{Idea de proyecto}


La propuesta de proyecto consiste en la resolución de ecuaciones de \textit{Euler-Lagrange} provenientes de la formulación de cálculo variacional, por medio de métodos espectrales.  


La idea es abordar las distintas ecuaciones \textit{EL} que aparecen en los distintos problemas (\textit{Image denoising, Image restoration, Image segmentation} entre otras) desde la forma más general posible.


Si denotamos los datos o función de entrada $f$, $E$ el funcional de energía que contiene las restricciones del problema y $\hat{u}$ la función  que minimiza $E(u)$, entonces la formulación variacional puede ser planteada como:
\begin{align*}
f &: \Omega \subset \mathbb{R}^n \rightarrow \mathbb{R} \\
\hat{u} &= \text{argmin} \ E(u) :=  \text{data}(u,f) + \alpha \ \text{smoothness}(u) + \beta \ \text{custom}(u),
\end{align*}
donde $\text{data}(\cdot)$ es la función que fuerza la similitud, $\text{smoothness}(\cdot)$ la función que fuerza la \textit{suavidad} de la función y $\text{custom}(\cdot)$ contiene las restricciones particulares del problema. Dada la formulación variacional, siempre es posible pasar a la formulación en ecuaciones diferenciales parciales (PDE), que consiste en encontrar una solución a $\mathbb{D}(u, \nabla u, \Delta u, \ldots)=0$, en donde el operador $\mathbb{D}$ computa la ecuación EL. Introduciendo una variable temporal, tal problema puede formularse como:
\begin{align*}
\partial_t u &= \mathbb{D}(u, \nabla u, \Delta u) \\
u(\cdot,0) &= f(\cdot) \ \ \ \ \in \Omega.
\end{align*}
Las derivadas espaciales en $\mathbb{D}(u, \nabla u, \Delta u)$ pueden ser aproximadas por matrices de diferenciación espectrales, y por lo tanto resolver el sistema dinámico debiera volverse computacionalmente eficiente.
\newline


Dentro del marco del proyecto, se propone también realizar un análisis comparativo con técnicas estándar en la resolución de estas PDE, tales como diferencias finitas y/o elementos finitos.
\end{document}
